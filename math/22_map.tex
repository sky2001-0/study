\lsection{写像}

\lsubsection{関係}

\dfn{二項関係}{
  集合$X, Y$について、集合系$G \in 2^{X \times Y}$を考える。
  以下のように定義されるアリティ2の述語記号を、順序対$\qty(\qty(X, Y), G)$によって特徴づけられた二項関係$R$と呼ぶ。
  \eq*{
    R(x, y) \defiff \qty(x, y) \in G
  }
  $R(x, y)$は誤解のない範囲において、$x R y$とも表す。

  また、二項関係は単に関係とも呼ぶ。
}

\cor{関係全体の集合}{
  集合$X, Y$間の関係の全体は集合をなす。具体的には\thmref{分出の公理図式}から以下のように書ける。
  \eq*{
    \qty{\qty(\qty(X, Y), G) \in \qty(\qty{X} \times \qty{Y}) \times 2^{X \times Y}}
  }
}

\dfn{始集合}{
  関係$R$について、定義より与えられる$X$を始集合または始域と呼ぶ。
}

\dfn{終集合}{
  関係$R$について、定義より与えられる$Y$を終集合または終域と呼ぶ。
}

\dfn{定義域}{
  \thmref{分出の公理図式}より定まる以下の集合を定義域$\dom(R)$と呼ぶ。
  \eq*{
    \dom(R) \coloneqq \qty{x \in X \mid \exists y \in Y \qty(R(x, y))}
  }
}

\dfn{値域}{
  関係$R$について、\thmref{分出の公理図式}より定まる以下の集合を値域$\ran(R)$と呼ぶ。
  \eq*{
    \ran(R) \coloneqq \qty{y \in Y \mid \exists x \in X \qty(R(x, y))}
  }
}

\dfn{逆}{
  関係$R$について、以下で定まる集合$G'$が一意に存在する。
  \eq*{
    G' = \qty{\qty(y, x) \in Y \times X \mid \qty(x, y) \in G}
  }

  $G'$を用いて定義される順序対$\qty(\qty(Y, X), G')$により特徴づけられる関係を、関係$R$の逆と呼び、$R^{-1}$と表す。
}

\dfn{左一意的}{
  関係$R$が左一意的であるとは、以下を満たすことである。
  \eq*{
    \forall w, x \in X \qty(\exists y \in Y \qty(R(w, y) \land R(x, y)) \rightarrow w = x)
  }
}

\dfn{右一意的}{
  関係$R$が右一意的であるとは、以下を満たすことである。
  \eq*{
    \forall y, z \in Y \qty(\exists x \in X \qty(R(x, y) \land R(x, z)) \rightarrow y = z)
  }
}

\dfn{一対一}{
  関係$R$が左一意的かつ右一意的であるとき、$R$は一対一であると言う。
}

\dfn{左全域的}{
  関係$R$が左全域的であるとは、以下を満たすことである。すなわち始集合と定義域が一致すことである。
  \eq*{
    \forall x \in X \exists y \in Y \qty(R(x, y))
  }
}

\dfn{右全域的}{
  関係$R$が右全域的であるとは、以下を満たすことである。すなわち終集合と値域が一致すことである。
  \eq*{
    \forall y \in Y \exists x \in X \qty(R(x, y))
  }
}

\dfn{対応}{
  関係$R$が左全域的かつ右全域的であるとき、$R$は対応であると言う。
}

\dfn{一対一対応}{
  関係$R$が一対一かつ対応であるとき、$R$は一対一対応であると言う。
}


\lsubsection{写像}

\dfn{部分写像}{
  右一意的な関係$R$について、以下の略記$f$を考える。
  \eq*{
    f(x) \coloneqq \bigcup \qty{y \in Y \mid R(x, y)}
  }

  右一意的であることから以下を満たす。
  \eq*{
    f(x) = y \leftarrow R(x, y)
  }

  この略記$f$を、関係$R$によって特徴づけられた部分写像と呼ぶ。

  始集合と終集合を明示したい場合、$f \colon X \nrightarrow Y$とも表す。
}

\dfn{写像}{
  右一意的かつ左全域的な関係$R$について、以下の略記$f$を考える。
  \eq*{
    f(x) \coloneqq \bigcup \qty{y \in Y \mid R(x, y)}
  }

  右一意的かつ左全域的であることから以下を満たす。
  \eq*{
    f(x) = y \leftrightarrow R(x, y)
  }

  この略記$f$を、関係$R$によって特徴づけられた写像と呼ぶ。

  始集合(すなわち定義域)と終集合を明示したい場合、$f \colon X \rightarrow Y$とも表す。

  関係を具体的に示したい場合、$f \colon X \ni x \mapsto f(x) \in Y$とも表す。

  また、$f(x)$を$x$での$f$の値と呼ぶ。
}

\cor{写像と空集合}{
  写像$f \colon X \rightarrow Y$について、左全域性より以下が成り立つ。
  \eq*{
    Y = \varnothing \rightarrow X = \varnothing
  }
}

\dfn{制限写像}{
  写像$f \colon X \rightarrow Y$と、$X$の部分集合$A$について以下の写像$\tilde{f}$を$f$の$A$への制限写像、または単に制限と呼ぶ。
  \eq*{
    f \rvert_A \colon A \ni x \mapsto f(x) \in Y
  }

  誤解のない限り、制限$f \rvert_A$を同様に$f$で表記する。
}

\dfn{写像全体の集合}{
  集合$X, Y$について、$X$を定義域、$Y$を終集合とする写像全体の集合が\corref{関係全体の集合}より存在して、$\Map(X, Y)$と表す。
}

\rem{集合の内包的定義3}{
  \axiref{置換の公理図式}の主張する$\psi(x, y)$について、関係$\qty(\qty(X, Y), \qty{\qty(x, y) \in X \times Y \mid \psi(x, y)})$より特徴づけられる関係$R$を考える。

  $R$は定義より右一意的である。$R$が左全域的であるとき、$R$により特徴づけられる写像$f$が存在して、\axiref{置換の公理図式}を満たす集合を以下のように定義する。

  \eq*{
    A = \qty{f(x) \mid x \in X}
  }
}

\dfn{恒等写像}{
  写像$f \colon X \ni \rightarrow x \in X$を恒等写像と呼ぶ。恒等写像であることを明示的に、$\id_{X}$と表す。
}

\dfn{合成写像}{
  写像$f \colon X \rightarrow Y, g \colon Y \rightarrow Z$について、以下で定める関係$R$を考える。
  \eqg*{
    G \coloneqq \qty{\qty(x, z) \in X \times Z \mid \exists y \in Y \qty(y = f(x) \land z = g(y))} \\*
    R \coloneqq \qty(\qty(X, Z), G)
  }

  この関係$R$は、右一意的かつ左全域的であるので、$R$によって特徴づけられる写像$h$が定まる。

  この写像$h$を$f$と$g$の合成写像と呼び、$g \circ f$と表す。
}

\cor{写像の合成の結合法則}{
  写像$f \colon X \rightarrow Y, g \colon Y \rightarrow Z, h \colon Z \rightarrow W$について、以下が成り立つ。
  \eq*{
    \qty(h \circ g) \circ f = h \circ \qty(g \circ f)
  }
}

\thm{選択公理が与える写像}{
  アリティ2の述語記号$\psi$について、$\forall x \in X \exists y \in Y \qty(\psi(x, y))$が成り立つとする。

  このとき、以下を満たす写像$f \colon X \rightarrow Y$が存在する。
  \eq*{
    \forall x \in X \forall y \in Y \qty(f(x) = y \rightarrow \psi(x, y))
  }
}{
  \axiref{置換の公理図式}より、以下の集合$Z$が存在する。
  \eq*{
    Z \coloneqq \qty{\qty{\qty(x, y) \mid y \in Y \land \psi(x, y)} \mid x \in X}
  }

  $Z$について\axiref{選択公理}の主張する集合$A$が存在する。順序対$\qty(\qty(X, Y), A)$の特徴づける関係は、仮定より左全域的で、$A$の定義より右一意的である。

  このような関係により特徴づけられる写像が存在する。
}

\dfn{像}{
  写像$f$の値域を、$f$の像と呼び、$\Im(f)$と表す。

  また、写像$f$の定義域$X$の部分集合$A$について、$A$の$f$による像$f(A)$を以下のように定める。
  \eq*{
    f(A) \coloneqq \qty{f(x) \mid x \in A}
  }
}

\thm{像の性質}{
  写像$f \colon X \rightarrow Y$について、
  \eqg*{
    \forall A_1, A_2 \qty(A_1 \subset A_2 \land A_2 \subset X \rightarrow f(A_1) \subset f(A_2)) \\*
    \forall B \in 2^X \qty(f(\bigcup B) = \bigcup \qty{f(A) \mid A \in B}) \\*
    \forall B \in 2^X \qty(f(\bigcap B) \subset \bigcap \qty{f(A) \mid A \in B}) \\*
    \forall A_1, A_2 \in 2^X \qty(f(A_1) \setminus f(A_2) \subset f(A_1 \setminus A_2))
  }
}{
  第一式について、
  $y \in f(A_1)$のとき、$\exists x \in A_1 \qty(y = f(x))$、$\exists x \in A_2 \qty(y = f(x))$、ゆえに$y \in f(A_2)$\\*

  第二式について、
  \eqg*{
    y \in \bigcup \qty{f(A) \mid A \in B} \\*
    \exists A \in B \qty(y \in f(A)) \\*
    \exists A \in B \exists x \in A \qty(y = f(x)) \\*
    \exists x \in \bigcup B \qty(y = f(x)) \\*
    y \in f(\bigcup B)
  }
  上からも下からも成り立つので、\axiref{外延性の公理}より成り立つ。\\*

  第三式について、第二式を用いて、
  \eqg*{
    y \in f(\bigcap B) \\*
    \exists x \qty(x \in \bigcap B \land y = f(x)) \\*
    \exists x \qty(x \in \bigcup B \land \forall A \qty(A \in B \rightarrow x \in A) \land y = f(x)) \\*
    \exists x \qty(x \in \bigcup B \land y = f(x)) \land \exists x \qty(\forall A \qty(A \in B \rightarrow x \in A) \land y = f(x)) \\*
    y \in f(\bigcup B) \land \forall A \qty(A \in B \rightarrow y \in f(A)) \\*
    y \in \bigcup \qty{f(A) \mid A \in B} \land \forall A \qty(A \in B \rightarrow y \in f(A)) \\*
    y \in \bigcap \qty{f(A) \mid A \in B}
  }\\*

  第四式について、
  \eqg*{
    y \in f(A_1) \setminus f(A_2) \\*
    \exists x \qty(y = f(x) \land x \in A_1) \land \forall x \qty(y = f(x) \rightarrow x \notin A_2) \\*
    \exists x \qty(y = f(x) \land x \in A_1 \setminus A_2) \\*
    y \in f(A_1 \setminus A_2)
  }
}

\dfn{原像}{
  写像$f \colon X \rightarrow Y$について、$Y$の部分集合$A$の原像とは、以下を満たす集合$f^{-1}(A)$である。
  \eq*{
    f^{-1}(A) = \qty{x \in X \mid f(x) \in A}
  }
}

\cor{像と原像}{
  写像$f \colon X \rightarrow Y$と$X$の部分集合$A$と$Y$の部分集合$B$について、
  \eqg*{
    f^{-1} \qty(f(A)) \supset A \\*
    f \qty(f^{-1}(B)) \subset B
  }
}

\thm{原像の性質}{
  写像$f \colon X \rightarrow Y$について、
  \eqg*{
    \forall A_1, A_2 \in 2^Y \qty(A_1 \subset A_2 \rightarrow f^{-1}(A_1) \subset f^{-1}(A_2)) \\*
    \forall B \in 2^Y \qty(f^{-1}(\bigcup B) = \bigcup \qty{f^{-1}(A) \mid A \in B}) \\*
    \forall B \in 2^Y \qty(f^{-1}(\bigcap B) = \bigcap \qty{f^{-1}(A) \mid A \in B}) \\*
    \forall A_1, A_2 \in 2^Y \qty(f^{-1}(A_1) \setminus f^{-1}(A_2) = f^{-1}(A_1 \setminus A_2))
  }
}{
  第一式について、
  $x \in f^{-1}(A_1)$のとき、$f(x) \in A_1 \subset A_2$ゆえに$x \in f^{-1}(A_2)$\\*

  第二式について、
  \eqg*{
    x \in \bigcup \qty{f^{-1}(A) \mid A \in B} \\*
    \exists A \in B \qty(x \in f^{-1}(A)) \\*
    \exists A \in B \qty(f(x) \in A) \\*
    f(x) \in \bigcup B \\*
    x \in f^{-1}(\bigcup B)
  }
  上からも下からも成り立つので、\axiref{外延性の公理}より成り立つ。\\*

  第三式について、第二式を用いて、
  \eqg*{
    x \in f^{-1}(\bigcap B) \\*
    f(x) \in \bigcap B \\*
    f(x) \in \bigcup B \land \forall A \qty(A \in B \rightarrow f(x) \in A) \\*
    x \in f^{-1}(\bigcup B) \land \forall A \qty(A \in B \rightarrow x \in f^{-1}(A)) \\*
    x \in \bigcup \qty{f^{-1}(A) \mid A \in B} \land \forall A \qty(A \in B \rightarrow x \in f^{-1}(A)) \\*
    x \in \bigcap \qty{f^{-1}(A) \mid A \in B}
  }
  上からも下からも成り立つので、\axiref{外延性の公理}より成り立つ。\\*

  第四式について、
  \eqg*{
    x \in f^{-1}(A_1) \setminus f^{-1}(A_2) \\*
    f(x) \in A_1 \land f(x) \notin A_2 \\*
    f(x) \in A_1 \setminus A_2 \\*
    x \in f^{-1}(A_1 \setminus A_2)
  }
  上からも下からも成り立つので、\axiref{外延性の公理}より成り立つ。
}


\lsubsection{単射と全射}

\dfn{単射、全射}{
  写像$f$が左一意的であるとき、$f$は単射であると言う。単射な写像を、誤解のない範囲で単射と呼ぶ。

  写像$f$が右全域的であるとき、$f$は全射であると言う。全射な写像を、誤解のない範囲で全射と呼ぶ。

  写像$f$が単射かつ全射であるとき、$f$は全単射であると言う。全単射な写像を、誤解のない範囲で全単射と呼ぶ。
}

\cor*{
  恒等写像は全単射である。
}

\thmf{\textit{Cantor}の対角線論法}{Cantorの対角線論法}{
  集合$A$について、単射$f \colon 2^A \rightarrow A$は存在しない。
}{
  存在するとする。

  以下の集合$Y$を考える。
  \eq*{
    Y \coloneqq \qty{f(X) \mid X \in 2^A \land f(X) \notin X}
  }
  このとき$Y \in 2^A$である。

  $f(Y) \notin Y$とすると、定義より$f(Y) \in Y$。ゆえに矛盾。

  $f(Y) \in Y$とすると、$f$の単射性より$Y \in 2^A \land f(Y) \notin Y$。ゆえに矛盾。

  背理法より示される。
}

\cor*{
  単射$f \colon X \rightarrow Y, g \colon Y \rightarrow Z$について、$g \circ f$は単射である。
}

\cor*{
  全射$f \colon X \rightarrow Y, g \colon Y \rightarrow Z$について、$g \circ f$は全射である。
}

\cor*{
  集合$X, Y$について、単射$X \rightarrow Y$が存在するならば、単射$2^X \rightarrow 2^Y$が存在する。
}

\cor*{
  集合$X, Y$について、全単射$2^{X \times Y} \rightarrow 2^X \times 2^Y$が存在する。
}

\dfn{左逆写像}{
  写像$f \colon X \rightarrow Y$について、以下を満たす写像$g \colon Y \rightarrow X$を左逆写像と呼ぶ。
  \eq*{
    g \circ f = \id_{X}
  }
}

\cor*{
  左逆写像は全射である。
}

\thm{単射と左逆写像}{
  空でない集合$X$と集合$Y$、写像$f \colon X \rightarrow Y$について、$f$が単射であることは、写像$f$が左逆写像$g$を持つことと同値である。
}{
  必要性を示す。

  空でないので$\exists x_0 \in X$より、以下のような関係$R$ができる。この$R$は、右一意的かつ対応である。
  \eqg*{
    G \coloneqq \qty{\qty(y, x) \in Y \times X \mid y = f(x)} \cup \qty{\qty(y, x_0) \mid y \in Y \setminus \Im(f)} \\*
    R \coloneqq \qty(\qty(Y, X), G)
  }

  関係$R$により特徴づけられる全射$g$は、定義より$f$の左逆写像となる。\\*

  十分性を示す。

  単射でないとすると、$\exists x, y \in X \qty(x \neq y \land f(x) = f(y))$で定義より、$x = g(f(x)) = g(f(y)) = y$より矛盾。背理法より示される。
}

\dfn{右逆写像}{
  写像$f \colon X \rightarrow Y$について、以下を満たす写像$g \colon Y \rightarrow X$を右逆写像と呼ぶ。
  \eq*{
    f \circ g = \id_{Y}
  }
}

\cor*{
  右逆写像は単射である。
}

\thm{全射と右逆写像}{
  集合$X, Y$、写像$f \colon X \rightarrow Y$について、$f$が全射であることは、写像$f$が右逆写像$g$を持つことと同値である。
}{
  必要性を示す。\thmref{選択公理が与える写像}の与える写像は、右逆写像である。\\*

  十分性を示す。

  全射でないとすると、$\exists y \in Y \forall x \in X \qty(f(x) \neq y)$で仮定に反する。背理法より示される。
}

\dfn{逆写像}{
  写像$f \colon X \rightarrow Y$について、写像$g \colon Y \rightarrow X$が$f$の左逆写像かつ右逆写像であるとき、逆写像と呼ぶ。
}

\cor*{
  逆写像は全単射である。
}

\thm{全単射と逆写像}{
  集合$X, Y$、写像$f \colon X \rightarrow Y$について、$f$が全単射であることは、写像$f$が逆写像$g$を持つことと同値である。
}{
  必要性を示す。$f$は全射より、$\Im(f) = Y$

  以下のような一対一対応$R$が存在する。
  \eqg*{
    G \coloneqq \qty{\qty(y, x) \in Y \times X \mid y = f(x)} \\*
    R \coloneqq \qty(\qty(Y, X), G)
  }

  関係$R$により、ある全単射$g$が特徴づけられ、定義より$f$の逆写像となる。\\*

  十分性を示す。

  $X$が空のとき、逆写像を持つので$Y$は空である。ゆえに全単射。

  $X$が空でないとき、\thmref{単射と左逆写像}と\thmref{全射と右逆写像}より$f$は全単射。
}

\lem*{
  写像$f \colon X \rightarrow Y$について、$f$の逆写像が存在するならば、一意に定まる。

  ここから、$f$の逆写像を$f^{-1}$と表す。
}{
  逆写像$g, h$について、$g(y) = h \circ f \circ g (y) =  h(y)$。ゆえに一意。
}

\cor*{
  全単射$f \colon X \rightarrow Y$について、
  \eq*{
    \qty(f^{-1})^{-1} = f
  }
}

\lem*{
  全単射$f \colon X \rightarrow Y$と$Y$の部分集合$A$について、原像$f^{-1}(A)$と逆写像の像$f^{-1}(A)$は一致する。
}{
  全単射より、$\forall y \in A \exists x \in A \qty(y = f(x))$。ゆえに定義より明らか。
}