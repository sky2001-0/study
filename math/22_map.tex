\lsection{写像}

\lsubsection{関係}

\dfn{二項関係}{
  集合系$G$について、以下のように定義されるアリティ2の述語記号$R_G$を、集合系$G$によって特徴づけられた二項関係と呼ぶ。
  \eq*{
    \forall x, y \qty(R_G(x, y) \defiff \qty(x, y) \in G)
  }
  $R_G(x, y)$は誤解のない範囲において、$x R_G y$とも表す。

  また、二項関係は単に関係とも呼ぶ。
}

\dfn{集合間の関係}{
  集合$X, Y, G$について、関係$R_G$が以下を満たすとき、$R_G$を$X$と$Y$の間の関係と呼ぶ。
  \eq*{
    G \in \P \qty(X, Y)
  }
}

\dfn{定義域}{
  \axiref{置換の公理図式}より定まる以下の集合を定義域$\dom(R_G)$と呼ぶ。
  \eq*{
    \dom(R_G) \coloneqq \qty{x \mid \qty(x, y) \in G}
  }
}

\dfn{値域}{
  関係$R$について、\axiref{置換の公理図式}より定まる以下の集合を値域$\ran(R_G)$と呼ぶ。
  \eq*{
    \ran(R_G) \coloneqq \qty{y \mid \qty(x, y) \in G}
  }
}

\cor*{
  任意の関係$R$は、$\dom(R)$と$\ran(R)$の間の関係である。
}

\dfn{逆}{
  関係$R_G$について、以下で定まる関係$R_\qty{G^{-1}}$が一意に存在する。
  \eq*{
    G^{-1} = \qty{\qty(y, x) \mid \qty(x, y) \in G}
  }
}

\dfn{左一意的}{
  関係$R$が左一意的であるとは、以下を満たすことである。
  \eq*{
    \forall w, x \qty(\exists y \qty(R(w, y) \land R(x, y)) \rightarrow w = x)
  }
}

\dfn{右一意的}{
  関係$R$が右一意的であるとは、以下を満たすことである。
  \eq*{
    \forall y, z \qty(\exists x \qty(R(x, y) \land R(x, z)) \rightarrow y = z)
  }
}

\dfn{一対一}{
  関係$R$が左一意的かつ右一意的であるとき、$R$は一対一であると言う。
}

\dfn{左全域的}{
  関係$R$が、集合$X$について左全域的であるとは、以下を満たすことである。
  \eq*{
    \forall x \in X \exists y \qty(R(x, y))
  }
}

\dfn{右全域的}{
  関係$R$が、集合$Y$について右全域的であるとは、以下を満たすことである。
  \eq*{
    \forall y \in Y \exists x \qty(R(x, y))
  }
}


\lsubsection{写像}

\dfn{部分写像}{
  集合$X, Y$について、$X$と$Y$の間の右一意的な関係$R_G$を考える。

  $x \in X$について$\qty{y \in Y \mid R_G(x, y)} \neq \varnothing$であるとき、以下の略記$f$を考える。
  \eq*{
    f(x) \coloneqq \bigcup \qty{y \in Y \mid R_G(x, y)}
  }

  右一意的であることから以下を満たす。
  \eq*{
    f(x) = y \leftrightarrow R_G(x, y)
  }

  この略記$f$を、順序対$\qty(\qty(X, Y), G)$によって特徴づけられた部分写像と呼ぶ。

  また、$f(x)$を$x$での$f$の値と呼ぶ。
}

\dfn{写像}{
  集合$X, Y$と、$X$と$Y$の間の右一意的な関係$R_G$について、$R_G$が$X$について左全域的であるとする。

  このとき、順序対$\qty(\qty(X, Y), G)$によって特徴づけられた部分写像$f$を写像と呼ぶ。
}

\dfn{写像の全体}{
  集合$X, Y$について、集合$Y^X$を以下のように定める。
  \eq*{
    Y^X \coloneqq \qty{\text{$\qty(\qty(X, Y), G)$の特徴づける写像} \mid G \in \P \qty(X \times Y) \land \text{$G$の特徴づける関係は、右一意的かつ$X$について左全域的}}
  }
}

\cor*{
  \eqg*{
    \forall Y \qty(Y^\varnothing = \qty{\varnothing}) \\*
    \forall X \qty(X \neq \varnothing \rightarrow \varnothing^X = \varnothing)
  }
}

\dfn{制限写像}{
  写像$f \in Y^X$と、$X$の部分集合$A$を考える。

  $f$を特徴づける集合系$G$について、以下の集合系$G_A$を考える。
  \eq*{
    G_A \coloneqq \qty{\qty(x, y) \in G \mid x \in A}
  }

  $\qty(\qty(A, Y), G_A)$に特徴づけられた写像$f \rvert_A$を$f$の$A$への制限写像、または単に制限と呼ぶ。

  誤解のない限り、制限$f \rvert_A$を同様に$f$で表記する。
}

\rem{集合の内包的定義3}{
  写像$f \in Y^X$について、以下を満たすアリティ2の述語記号$\psi(x, y)$を考える。
  \eq*{
    \forall x, y \qty(\psi(x, y) \defiff y = f(x))
  }

  $f$が写像であることから\axiref{置換の公理図式}の主張する集合$A$が存在する。これをより簡潔に以下のように表すものとする。
  \eq*{
    A = \qty{f(x) \mid x \in X}
  }
}

\dfn{恒等写像}{
  集合$X$について、以下で定める集合系$\Delta$を考える。$\qty(\qty(X, X), \Delta)$により特徴づけられる写像$\id_X$を恒等写像と呼ぶ。
  \eq*{
    \Delta \coloneqq \qty{\qty(x, x) \mid x \in X}
  }
}

\dfn{合成写像}{
  写像$f \in Y^X, g \in Z^Y$について、以下で定める集合系$G$を考える。
  \eq*{
    G \coloneqq \qty{\qty(x, z) \mid \exists y \in Y \qty(y = f(x) \land z = g(y))}
  }

  $\qty(\qty(X, Z), G)$によって特徴づけられる写像$h$が存在する。

  この写像$h$を$f$と$g$の合成写像と呼び、$g \circ f$と表す。
}

\cor{写像の合成の結合法則}{
  写像$f \in Y^X, g \in Z^Y, h \in W^Z$について、以下が成り立つ。
  \eq*{
    \qty(h \circ g) \circ f = h \circ \qty(g \circ f)
  }
}

\thm{選択公理が与える写像}{
  アリティ2の述語記号$\psi$について、$\forall x \in X \exists y \in Y \qty(\psi(x, y))$が成り立つとする。

  このとき、以下を満たす写像$f \in Y^X$が存在する。
  \eq*{
    \forall x \in X \forall y \in Y \qty(f(x) = y \rightarrow \psi(x, y))
  }
}{
  以下の集合$Z$を考える。
  \eq*{
    Z \coloneqq \qty{\qty{\qty(x, y) \mid y \in Y \land \psi(x, y)} \mid x \in X}
  }

  $Z$について\axiref{選択公理}の主張する集合$A$が存在する。$A$の特徴づける関係は、仮定より$X$について左全域的で、$A$の定義より右一意的である。

  $\qty(\qty(X, Y), A)$により特徴づけられる写像$f$が存在する。
}

\dfn{像}{
  写像$f \in Y^X$の値域を、$f$の像と呼び、$\Im(f)$と表す。

  また、$X$の部分集合$A$について、$A$の$f$による像$f(A)$を以下のように定める。
  \eq*{
    f(A) \coloneqq \qty{f(x) \mid x \in A}
  }
}

\thm{像の性質}{
  写像$f \in Y^X$について、以下が成り立つ。
  \eqg*{
    \forall A_1, A_2 \in \P(X) \qty(A_1 \subset A_2 \rightarrow f(A_1) \subset f(A_2)) \\*
    \forall B \in \P \qty(\P(X)) \qty(f \qty(\bigcup B) = \bigcup \qty{f(A) \mid A \in B}) \\*
    \forall B \in \P \qty(\P(X)) \qty(f \qty(\bigcap B) \subset \bigcap \qty{f(A) \mid A \in B}) \\*
    \forall A_1, A_2 \in \P(X) \qty(f(A_1) \setminus f(A_2) \subset f(A_1 \setminus A_2))
  }
}{
  第一式について、
  $y \in f(A_1)$のとき、$\exists x \in A_1 \qty(y = f(x))$、$\exists x \in A_2 \qty(y = f(x))$、ゆえに$y \in f(A_2)$\\*

  第二式について、
  \eqg*{
    y \in \bigcup \qty{f(A) \mid A \in B} \\*
    \exists A \in B \qty(y \in f(A)) \\*
    \exists A \in B \exists x \in A \qty(y = f(x)) \\*
    \exists x \in \bigcup B \qty(y = f(x)) \\*
    y \in f(\bigcup B)
  }
  上からも下からも成り立つので、\axiref{外延性の公理}より成り立つ。\\*

  第三式について、第二式を用いて、
  \eqg*{
    y \in f(\bigcap B) \\*
    \exists x \qty(x \in \bigcap B \land y = f(x)) \\*
    \exists x \qty(x \in \bigcup B \land \forall A \qty(A \in B \rightarrow x \in A) \land y = f(x)) \\*
    \exists x \qty(x \in \bigcup B \land y = f(x)) \land \exists x \qty(\forall A \qty(A \in B \rightarrow x \in A) \land y = f(x)) \\*
    y \in f(\bigcup B) \land \forall A \qty(A \in B \rightarrow y \in f(A)) \\*
    y \in \bigcup \qty{f(A) \mid A \in B} \land \forall A \qty(A \in B \rightarrow y \in f(A)) \\*
    y \in \bigcap \qty{f(A) \mid A \in B}
  }\\*

  第四式について、
  \eqg*{
    y \in f(A_1) \setminus f(A_2) \\*
    \exists x \qty(y = f(x) \land x \in A_1) \land \forall x \qty(y = f(x) \rightarrow x \notin A_2) \\*
    \exists x \qty(y = f(x) \land x \in A_1 \setminus A_2) \\*
    y \in f(A_1 \setminus A_2)
  }
}

\dfn{原像}{
  写像$f$について、集合$A$の原像とは、以下を満たす集合$f^{-1}(A)$である。
  \eq*{
    f^{-1}(A) = \qty{x \in X \mid f(x) \in A}
  }
}

\cor{像と原像}{
  写像$f \in Y^X$、$X$の部分集合$A$、集合$B$について、
  \eqg*{
    f^{-1} \qty(f(A)) \supset A \\*
    f \qty(f^{-1}(B)) \subset B
  }
}

\thm{原像の性質}{
  写像$f \in Y^X$について、以下が成り立つ。
  \eqg*{
    \forall A_1, A_2 \in \P(Y) \qty(A_1 \subset A_2 \rightarrow f^{-1}(A_1) \subset f^{-1}(A_2)) \\*
    \forall B \in \P \qty(\P(Y)) \qty(f^{-1}(\bigcup B) = \bigcup \qty{f^{-1}(A) \mid A \in B}) \\*
    \forall B \in \P \qty(\P(Y)) \qty(f^{-1}(\bigcap B) = \bigcap \qty{f^{-1}(A) \mid A \in B}) \\*
    \forall A_1, A_2 \in \P(Y) \qty(f^{-1}(A_1) \setminus f^{-1}(A_2) = f^{-1}(A_1 \setminus A_2))
  }
}{
  第一式について、
  $x \in f^{-1}(A_1)$のとき、$f(x) \in A_1 \subset A_2$ゆえに$x \in f^{-1}(A_2)$\\*

  第二式について、
  \eqg*{
    x \in \bigcup \qty{f^{-1}(A) \mid A \in B} \\*
    \exists A \in B \qty(x \in f^{-1}(A)) \\*
    \exists A \in B \qty(f(x) \in A) \\*
    f(x) \in \bigcup B \\*
    x \in f^{-1}(\bigcup B)
  }
  上からも下からも成り立つので、\axiref{外延性の公理}より成り立つ。\\*

  第三式について、第二式を用いて、
  \eqg*{
    x \in f^{-1}(\bigcap B) \\*
    f(x) \in \bigcap B \\*
    f(x) \in \bigcup B \land \forall A \qty(A \in B \rightarrow f(x) \in A) \\*
    x \in f^{-1}(\bigcup B) \land \forall A \qty(A \in B \rightarrow x \in f^{-1}(A)) \\*
    x \in \bigcup \qty{f^{-1}(A) \mid A \in B} \land \forall A \qty(A \in B \rightarrow x \in f^{-1}(A)) \\*
    x \in \bigcap \qty{f^{-1}(A) \mid A \in B}
  }
  上からも下からも成り立つので、\axiref{外延性の公理}より成り立つ。\\*

  第四式について、
  \eqg*{
    x \in f^{-1}(A_1) \setminus f^{-1}(A_2) \\*
    f(x) \in A_1 \land f(x) \notin A_2 \\*
    f(x) \in A_1 \setminus A_2 \\*
    x \in f^{-1}(A_1 \setminus A_2)
  }
  上からも下からも成り立つので、\axiref{外延性の公理}より成り立つ。
}

\dfn{一般の直積集合}{
  写像$A \in Z^\Lambda$について、以下を満たす集合を写像$A$の直積集合と呼ぶ。

  \eq*{
    \prod A \coloneqq \qty{f \in \qty(\bigcup \qty{A(\lambda) \mid \lambda \in \Lambda})^\Lambda \mid \forall \lambda \in \Lambda \qty(f(\lambda) \in A(\lambda))}
  }
}

\cor*{
  集合$X, Y$について、写像$A \in \qty{Y}^X, \forall x \in X \qty(A(x) \coloneqq Y)$を考える。

  このとき、$\prod A = Y^X$である。
}


\lsubsection{単射と全射}

\dfn{単射}{
  写像$f$が左一意的であるとき、$f$は単射であると言う。
  単射な写像を、誤解のない範囲で単射と呼ぶ。
}

\cor*{
  単射$f \in Y^X, g \in Z^Y$について、$g \circ f$は単射である。
}

\cor*{
  集合$X, Y$について、単射$f \in Y^X$が存在するならば、単射$g \in \P(Y)^{\P(X)}$が存在する。
}

\dfn{全射}{
  写像$f$が右全域的であるとき、$f$は全射であると言う。
  全射な写像を、誤解のない範囲で全射と呼ぶ。
}

\cor*{
  全射$f \in Y^X, g \in Z^Y$について、$g \circ f$は全射である。
}

\dfn{全単射}{
  写像$f$が単射かつ全射であるとき、$f$は全単射であると言う。
  全単射な写像を、誤解のない範囲で全単射と呼ぶ。
}

\cor*{
  恒等写像は全単射である。
}

\dfn{左逆写像}{
  写像$f \in Y^X$について、以下を満たす写像$g \in X^Y$を左逆写像と呼ぶ。
  \eq*{
    g \circ f = \id_{X}
  }
}

\cor*{
  左逆写像は全射である。
}

\thm{単射と左逆写像}{
  空でない集合$X$と集合$Y$、写像$f \in Y^X$について、$f$が単射であることは、写像$f$が左逆写像$g$を持つことと同値である。
}{
  必要性を示す。

  空でないので$\exists x_0 \in X$より、以下で定める集合$G$が存在する。
  \eq*{
    G \coloneqq \qty{\qty(y, x) \in Y \times X \mid y = f(x)} \cup \qty{\qty(y, x_0) \mid y \in Y \setminus \Im(f)}
  }

  $\qty(\qty(X, Y), G)$により特徴づけられる全射$g$は、定義より$f$の左逆写像となる。\\*

  十分性を示す。

  単射でないとすると、$\exists x, y \in X \qty(x \neq y \land f(x) = f(y))$で定義より、$x = g(f(x)) = g(f(y)) = y$より矛盾。背理法より示される。
}

\dfn{右逆写像}{
  写像$f \in Y^X$について、以下を満たす写像$g \in X^Y$を右逆写像と呼ぶ。
  \eq*{
    f \circ g = \id_{Y}
  }
}

\cor*{
  右逆写像は単射である。
}

\thm{全射と右逆写像}{
  集合$X, Y$、写像$f \in Y^X$について、$f$が全射であることは、写像$f$が右逆写像$g$を持つことと同値である。
}{
  必要性を示す。\thmref{選択公理が与える写像}の与える写像は、右逆写像である。\\*

  十分性を示す。

  全射でないとすると、$\exists y \in Y \forall x \in X \qty(f(x) \neq y)$で仮定に反する。背理法より示される。
}

\dfn{逆写像}{
  写像$f \in Y^X$について、写像$g \in X^Y$が$f$の左逆写像かつ右逆写像であるとき、逆写像と呼ぶ。
}

\cor*{
  逆写像は全単射である。
}

\thm{全単射と逆写像}{
  集合$X, Y$、写像$f \in Y^X$について、$f$が全単射であることは、写像$f$が逆写像$g$を持つことと同値である。
}{
  必要性を示す。$f$は全射より、$\Im(f) = Y$

  以下のような集合系$G$が存在する。
  \eq*{
    G \coloneqq \qty{\qty(y, x) \in Y \times X \mid y = f(x)}
  }

  $\qty(\qty(X, Y), G)$により特徴づけられる写像$g$は定義より$f$の逆写像となる。\\*

  十分性を示す。

  $X$が空のとき、逆写像を持つので$Y$は空である。ゆえに全単射。

  $X$が空でないとき、\thmref{単射と左逆写像}と\thmref{全射と右逆写像}より$f$は全単射。
}

\lem*{
  写像$f \in Y^X$について、$f$の逆写像が存在するならば、一意に定まる。

  ここから、$f$の逆写像を$f^{-1}$と表す。
}{
  逆写像$g, h$について、$g(y) = h \circ f \circ g (y) =  h(y)$。ゆえに一意。
}

\cor*{
  全単射$f \in Y^X$について、
  \eq*{
    \qty(f^{-1})^{-1} = f
  }
}

\lem*{
  全単射$f \in Y^X$と$Y$の部分集合$A$について、原像$f^{-1}(A)$と逆写像の像$f^{-1}(A)$は一致する。
}{
  全単射より、$\forall y \in A \exists x \in A \qty(y = f(x))$。ゆえに定義より明らか。
}

\thmf{\textit{Cantor}の対角線論法}{Cantorの対角線論法}{
  集合$A$について、単射$f \in A^{\P(A)}$は存在しない。
}{
  存在するとする。

  以下の集合$Y$を考える。
  \eq*{
    Y \coloneqq \qty{f(X) \mid X \in \P(A) \land f(X) \notin X}
  }
  このとき$Y \in \P(A)$である。

  $f(Y) \notin Y$とすると、定義より$f(Y) \in Y$。ゆえに矛盾。

  $f(Y) \in Y$とすると、$f$の単射性より$Y \in \P(A) \land f(Y) \notin Y$。ゆえに矛盾。

  背理法より示される。
}