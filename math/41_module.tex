\lsection{加群}

\lsubsection{加群}

\dfn{左加群}{
  環$R$、可換群$\qty(V, +)$、\thmref{可換群上の自己準同型全体}の主張する環$\End(V)$について、環準同型$\rho \in \End(V)^R$が存在するとき、順序対$\qty(\qty(V, +), \rho)$を左$R$-加群と呼ぶ。
  または単に$V$と書き、左$R$-加群を表すものとする。

  作用の値$\rho(r)(x)$を$r x$と略記する。
}

\rem{左加群の補足}{
  左$R$-加群$\qty(\qty(V, +), \rho)$は以下を満たす。$\forall a, b, c \in V \forall r, s \in R$とする。
  \begin{enumerate}
    \item $V$のマグマ性:$a + b \in V$
    \item $V$の半群性:$a + \qty(b + c) = \qty(a + b) + c$
    \item $V$のモノイド性:$\exists 0_V \coloneqq e$
    \item $V$の群性:$\exists (-a)$
    \item $V$の可換群性:$a + b = b + a$
    \item 自己準同型性:$r \qty(a + b) = r a + r b$
    \item $\rho$の群準同型性と$\End(V)$の定義:$\qty(r + s) a = r a + s a$
    \item $\rho$のモノイド準同型性:$r \qty(s a) = \qty(r s) a$
  \end{enumerate}
}

\dfn{加群}{
  左$R$-加群について、$R$が可換環であるとき、$R$-加群と呼ぶ。
}

\lem{加群の性質}{
  左$R$-加群$V$について、以下が成り立つ。
  \eqg*{
    \forall v \in V \qty(0_R v = 0_V) \\*
    \forall r \in R \forall v \in V \qty(r (-v) = (-r) v = -(r v)) \\*
    \forall r \in R \forall v \in V \qty(r v = (-r) (-v))
  }
}{
  \lemref{環の性質}と同様に示される。
}

\dfn{線型}{
  左$R$-加群$V_1, V_2$について、群準同型$f \in V_2^{V_1}$が以下を満たすとき、$f$を加群準同型写像または線型写像と呼ぶ。
  \eq*{
    \forall v \in V_1 \forall r \in R \qty(f(r v) = r f(v))
  }
}

\dfn{加群同型}{
  全単射な線型写像を、加群同型写像、または単に加群同型と呼ぶ。
}

\dfn{線型写像の全体}{
  左$R$-加群$V, V'$について、$V$から$V'$への線型写像全体のなす集合は群をなす。
  これを$\Hom_R(V, V')$と表す。

  さらに$\End_R(V) \coloneqq \Hom_R(V, V)$、$\GL(V) \coloneqq \End_R(V)^\times$とする。
}

\thm{左加群としての環}{
  環$\qty(\qty(R, +), \times)$と、$R$の部分環$S$を考える。

  以下で定める$\rho \in \End(R)^S$について、順序対$\qty(\qty(R, +), \rho)$は左$S$-加群である。
  \eq*{
    \forall s \in S \forall r \in R \qty(\rho(s) (r) \coloneqq s \times r)
  }
}{
  環の分配法則、および積についての結合性から、$\rho$は環準同型である。
}

\thmf{$\Z$-加群としての可換群}{Z-加群としての可換群}{
  可換群$\qty(G, +)$を考える。以下で定める$\rho \in \End(G)^\Z$について、順序対$\qty(\qty(G, +), \rho)$は$\Z$-加群である。
  \eq*{
    \rho(n)(g) \coloneqq
    \begin{cases}
      \sum_{m \in n} g & \qty(0 \leq n) \\*
      \sum_{m \in (-n)} (-g) & \qty(n < 0)
    \end{cases}
  }
}{
  正負の場合の場合分けを考えて、\lemref{モノイド上の指数法則}より、成り立つ。
}


\lsubsection{加群と準同型定理}

\dfn{部分加群}{
  左$R$-加群$\qty(\qty(V, +), \rho)$と$V$の部分群$\qty(W, +)$について、以下が成り立つとき$\qty(\qty(W, +), \rho')$を$\qty(\qty(V, +), \rho)$の部分加群と呼ぶ。
  \eq*{
    \forall v \in W \forall r \in R \qty(r v \in W)
  }
}

\dfn{商加群}{
  左$R$-加群$\qty(\qty(V, +), \rho)$上の同値関係$\sim$を考える。

  \corref{直積集合と自明な同値関係}の意味で演算$+$と$\sim$が両立かつ、$\forall r \in R$について写像$\rho(r)$が$\sim$と両立するとき、

  \thmref{両立}より定める演算$+'$および環準同型$\rho'$が存在する。

  このとき、左$R$-加群$\qty(\qty(V / \sim, +'), \rho')$を左$R$-加群$V$の商加群と呼ぶ。
}

\lem*{
  左$R$-加群$V$とその部分加群$W$について、以下で定める関係$\sim$は\corref{直積集合と自明な同値関係}の意味で演算$+$と両立かつ、$\forall r \in R$について写像$\rho(r)$が$\sim$と両立する同値関係である。
  \eq*{
    \forall x, y \in V \qty(x \sim y \defiff x + (-y) \in W)
  }
}{
  \lemref{部分群の定める同値関係}より、$+$と両立する同値関係である。

  $x + (-y) \in W$のとき、\lemref{加群の性質}より、$\forall r \in R \qty(r x + (-(r y)) = r x + r (-y) = r \qty(x + (-y)) \in W)$

  ゆえに両立する。
}

\dfn{剰余加群}{
  左$R$-加群$V$とその部分加群$W$について、\mlemref{0}の定める同値関係による剰余群を、剰余加群と呼び、$V / W$と表す。
}

\thm{加群の準同型定理}{
  左$R$-加群$V_1, V_2$と、線型写像$f \in V_2^{V_1}$に付随する同値関係$\sim_f$について、加群同型$\bar{f} \in \Im(f)^{V / \sim_f}$が存在する。
}{
  $\Im(f)$は$V_2$の部分左加群である。

  \thmref{群準同型定理}より、得る$\bar{f}$は群同型である。

  今、$\bar{f}(\qty[r v]) = f(r v) = r f(v) = r \bar{f}([v])$であるので線型である。
}


\lsubsection{自由加群}

\dfn{線型包}{
  左$R$-加群$V$について、$V$の部分集合$W$を考える。このとき、以下で定める集合$\Span W$は$V$の部分加群である。
  \eq*{
    \Span W \coloneqq \qty{v \in V \mid \exists n \in \N \exists \qty(r_m, w_m)_{m \in \N} \in \qty(R \times W)^\N \qty(v = \sum_{m \in n} r_m w_m)}
  }

  これを線型包と呼ぶ。
}

\dfn{生成系}{
  左$R$-加群$V$について、$V$の部分集合$W$が以下を満たすとき、$W$を$V$の生成系と呼ぶ。
  \eq*{
    V = \Span W
  }
}

\dfn{一次独立}{
  左$R$-加群$V$と、$V$の部分集合$W$が以下を満たすとき、$W$は一次独立であると呼ぶ。
  \eq*{
    \forall n \in \N \forall \qty(r_m, w_m)_{m \in \N} \in \qty(R \times W)^\N \qty(\sum_{m \in n} r_m w_m = 0_V \rightarrow \forall m \in n \qty(r_m = 0_R))
  }
}

\lem{一次独立と一意性}{
  左$R$-加群$V$と、$V$の一次独立な部分集合$W$について、以下が成り立つ。

  $\exists n_1, n_2 \in \N \exists \qty(r_{1, m}, w_{1, m})_{m \in \N}, \qty(r_{2, m'}, w_{2, m'})_{m' \in \N} \in \qty(R \times W)^\N$について、
  \eq*{
    \sum_{m \in n_1} r_{1, m} w_{1, m} = \sum_{m' \in n_2} r_{2, m'} w_{2, m'} \rightarrow \forall m \in n_1 \qty(r_{1, m} = 0_R \lor \exists m' \in n_2 \qty(r_{1, m} = r_{2, m'} \land w_{1, m} = w_{2, m'}))
  }
}{
  $0_V = \sum_{m \in n_1} r_{1, m} w_{1, m} = \sum_{m' \in n_2} r_{2, m'} w_{2, m'}$より、一次独立の定義から成り立つ。
}

\dfn{基底}{
  左$R$-加群$V$と$V$の部分集合$B$について、$B$が一次独立かつ$V$の生成系であるとき、$B$を$V$の基底と呼ぶ。
}

\dfn{自由加群}{
  基底を持つ左$R$-加群を自由加群と呼ぶ。
}


\lsubsection{加群の双対}

\lem*{
  可換環$R$と、$R$-加群$\qty(V, \rho)$について、以下で定める順序対$\qty(\Hom_R(V, R), \rho')$は$R$-加群である。
  \eq*{
    \forall f \in \Hom_R(V, R) \forall r \in R \forall v \in V \qty(\rho'(r)(f) (v) \coloneqq f \circ \rho(r)(v))
  }
}{
  \thmref{可換群上の準同型全体}より、$\Hom_R(V, R)$は可換群である。

  $\rho$の環準同型性より、以下が成り立つ。
  \eqa*{
    \rho'(r + s)(f) (v)
    &= f \circ \rho(r + s)(v) \\*
    &= f \qty(\rho(r)(v) + \rho(s)(v)) \\*
    &= f \circ \rho(r) (v) + f \circ \rho(s) (v) \\*
    &= \qty(\rho'(r)(f) + \rho'(r)(g)) (v)
  }

  \eqa*{
    \rho'(s r)(f) (v)
    &= \rho'(r s)(f) (v) \\*
    &= f \circ \rho(r s)(v) \\*
    &= f \circ \rho(r) \circ \rho(s) (v) \\*
    &= \rho'(r) (f) \circ \rho(s) (v) \\*
    &= \rho'(s) \circ \rho'(r) (f) (v)
  }
}

\dfn{双対加群}{
  \mlemref{0}より定まる加群を$V$の双対と呼び、$V^*$で表す。
}

\dfn{標準双線型式}{
  $R$-加群$V$について、以下で定まる写像$\braket{} \in R^{V^* \times V}$を標準双線形式と呼ぶ。
  \eq*{
    \forall \qty(f, x) \in V^* \times V \qty(\braket{f}{x} \coloneqq f(x))
  }
}

\rem{双対加群の補足}{
  $R$-加群$V$は以下を満たす。$\forall x, y \in V \forall f, g \in V^* \forall r \in R$とする。
  \begin{enumerate}
    \item $f$の線型性:$\braket{f}{x + y} = \braket{f}{x} + \braket{f}{y}$
    \item $f$の線型性:$\braket{f}{r x} = r \braket{f}{x}$
    \item \thmref{可換群上の準同型全体}:$\braket{f + g}{x} = \braket{f}{x} + \braket{g}{x}$
    \item 双対加群の定義:$\braket{r f}{x} = r \braket{f}{x}$
  \end{enumerate}
}

\dfn{双対写像}{
  $R$-加群$V, W$を考える。線型写像$u \in W^V$について、以下で定まる写像$u^* \in \qty(V^*)^{W^*}$を双対写像と呼ぶ。
  \eq*{
    \forall f \in W^* \qty(u^*(f) \coloneqq f \circ u)
  }

  ゆえに、$\braket{f}{u(x)} = \braket{u^*(f)}{v}$が直ちに求まる。
}


\lsubsection{加群の積}

\dfn{直積加群}{
  環$R$と、写像$V \in Z^\Lambda$を考える。$\forall \lambda \in \Lambda$について、$V(\lambda)$は左$R$-加群であるとする。

  以下で定める順序対$\qty(\qty(\prod V, +), \rho)$は左加群である。
  \eqg*{
    \forall v_1, v_2 \in \prod V \forall \lambda \in \Lambda \qty(\qty(v_1 + v_2)(\lambda) \coloneqq v_1(\lambda) + v_2(\lambda)) \\*
    \forall v \in \prod V \forall r \in R \forall \lambda \in \Lambda \qty(\qty(r v)(\lambda) \coloneqq r v(\lambda))
  }

  この左$R$-加群を$V$の直積加群または直積と呼ぶ。
}

\dfn{直和加群}{
  環$R$と、写像$V \in Z^\Lambda$を考える。$\forall \lambda \in \Lambda$について、$V(\lambda)$は左$R$-加群であるとする。

  以下で定める順序対$\qty(\qty(\bigoplus V, +), \rho)$は左加群である。
  \eq*{
    \bigoplus V \coloneqq \qty{v \in \prod V \mid \abs{\qty{\lambda \in \Lambda \mid v(\lambda) \neq 0_{V(\lambda)}}} < \infty}
  }

  この左$R$-加群を$V$の直積加群または直和と呼ぶ。
}

\cor*{
  直和$\bigoplus V$は、直積$\prod V$の部分加群である。
}

\cor*{
  $V \in Z^\Lambda$を考える。$\forall \lambda \in \Lambda$について、$V(\lambda)$は左$R$-加群であるとする。

  $\abs{\Lambda} < \infty$ならば$\bigoplus V = \prod V$である。
}

\lem*{
  可換環$R$と、$R$-加群$V_1, V_2$、写像$A \in \qty{R}^{V_1 \times V_2}$を考える。

  以下で定める単射$\varphi \in \qty(\bigoplus A)^{V_1 \times V_2}$を考える。
  \eq*{
    \varphi \qty(x, y) \qty(x', y') =
    \begin{cases}
      1_R & \qty(x = x' \land y = y') \\*
      0_R & \qty(\otherwise)
    \end{cases}
  }

  さらに以下で定める$B$は、$\bigoplus A$の部分加群である。
  \eqg*{
    B_1 \coloneqq \qty{\varphi \qty(x_1 + x_2, y) - \varphi \qty(x_1, y) - \varphi \qty(x_2, y) \mid x_1, x_2 \in V_1, y \in V_2} \\*
    B_2 \coloneqq \qty{\varphi \qty(x, y_1 + y_2) - \varphi \qty(x, y_1) - \varphi \qty(x, y_2) \mid x \in V_1, y_1, y_2 \in V_2} \\*
    B_3 \coloneqq \qty{r \varphi \qty(x, y) - \varphi \qty(r x, y) \mid r \in R, x \in V_1, y \in V_2} \\*
    B_4 \coloneqq \qty{r \varphi \qty(x, y) - \varphi \qty(x, r y) \mid r \in R, x \in V_1, y \in V_2} \\*
    B \coloneqq \Span \qty(\bigcup \qty{B_1, B_2, B_3, B_4})
  }
}{
  $\bigcup \qty{B_1, B_2, B_3, B_4} \subset \bigoplus A$より明らか。
}

\dfn{テンソル積}{
  可換環$R$と、$R$-加群$V_1, V_2$、写像$A \in \qty{R}^{V_1 \times V_2}$を考える。

  \mlemref{0}の主張する$R$-加群$B$について、剰余加群$\qty(\bigoplus A) / B$をテンソル積と呼び、$V_1 \bigotimes V_2$と表す。

  さらに\mlemref{0}の定める$\varphi$について、$\qty[\varphi \qty(x, y)]$を$x \otimes y$と表す。
}

\rem{テンソル積の補足}{
  可換環$R$と、$R$-加群$V_1, V_2$は以下を満たす。$\forall x, x_1, x_2 \in V_1 \forall y, y_1, y_2 \in V_2 \forall r \in R$とする。
  \begin{enumerate}
    \item $\qty(x_1 + x_2) \otimes y = x_1 \otimes y + x_2 \otimes y$
    \item $x \otimes \qty(y_1 + y_2) = x \otimes y_1 + x \otimes y_2$
    \item $r \qty(x \otimes y) = \qty(r x) \otimes y = x \otimes \qty(r y)$
  \end{enumerate}
}

\begin{comment}

\dfn{テンソル空間}{}

\lsubsection{外積代数}

\dfn{外積}{}

\dfn{行列式}{}

\dfn{準双線型写像}{}

\dfnf{\textit{Hermite}}{Hermite}{}

\end{comment}