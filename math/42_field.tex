\lsection{体論}

\lsubsection{体}

\dfn{体}{
  可換環$\qty(\qty(F, +), \times)$が以下を満たすとき、体と呼ぶ。
  \eq*{
    F^\times = F \setminus \qty{0_F}
  }

  さらに、$x \times y^{-1}$を$x / y$と略記する。
}

\cor*{
  体は零環ではない。
}

\dfn{部分体}{
  体$F$の部分環$R$が、体をなすとき、体$R$を体$F$の部分体と呼ぶ。
}

\thm{体はED}{
  体$F$は\textit{Euclid}整域である。
}{
  まず、整域であることを示す。

  整域でないとすると、$\forall x, y \in F \qty(x y = 0_F \land x \neq 0_F \land y \neq 0_F)$である。

  $y = x^{-1} x y = x^{-1} 0_F = 0_F$より矛盾。ゆえに整域。

  $\forall a \in F \forall b \in D \setminus \qty{0_D}$について、$q = a b^{-1} \land r = 0_D$として成り立つ。
}

\thm{極大イデアルと体}{
  可換環$R$とイデアル$I$について、以下の2つは同値である。
  \begin{enumerate}
    \item $I$は極大イデアル
    \item 剰余環$R / I$は体
  \end{enumerate}
}{
  $1. \rightarrow 2.$を示す。

  $\forall a \in R \qty(\qty[a] \neq 0_{R / I})$を考える。

  $a \notin I$である。ここで、$J \coloneqq \qty{r a + i \mid \qty(r, i) \in R \times I}$を考える。

  定義より$I \subset J$であり、$J$はイデアル。ゆえに$I$の極大性から$J = R$

  したがって$\exists \qty(r, i) \in R \times I \qty(r a + i = 1_R)$である。

  よって$1_{R / I} = \qty[1_R] = \qty[r a + i] = \qty[r] \qty[a] + \qty[i] = \qty[r] \qty[a]$より、$\qty[r] = \qty[a]^{-1}$\\*

  $2. \rightarrow 1.$を示す。

  $I$が極大イデアルでないとする。すなわちイデアル$J$が存在して、$I \subsetneq J \subsetneq R$である。

  $\forall a \in J \setminus I$について、$\qty[a] \neq 0_{R / I}$である。

  体であるので、$\exists b \in R \qty(\qty[b] \qty[a] = \qty[1_R])$である。

  したがって$b a - 1_R \in I \subset J$である。ここで、$a \in J$より$b a \in J$

  イデアルは加法について群をなすので$1_R = b a - \qty(b a - 1_R) \in J$である。したがって$J = R$となり矛盾。
}

\thm{体準同型は単射}{
  体$F$、零環ではない環$R$について、環準同型$\varphi \in R^F$は単射である。
}{
  $\qty{0_F}$は、\thmref{極大イデアルと体}より、極大イデアルである。

  $\Ker(\varphi)$はイデアルであり、$\qty{0_F} \subset \Ker(\varphi)$である。

  ゆえに、$\Ker(\varphi) = \qty{0_F} \lor \Ker(\varphi) = F$

  零環でない$R$について、$\varphi(1_F) = 1_R \neq 0_R$より、$1_R \notin \Ker(\varphi)$

  したがって、$\Ker(\varphi) = \qty{0_F}$

  \thmref{群準同型の単射性}より単射。
}


\lsubsection{体上の加群}

\thm{基底の存在}{
  体$F$について、$\qty{0_V}$でない$F$-加群$V$は自由加群である。
}{
  $\mathcal{S} \coloneqq \qty{S \in \P(V) \mid \text{$S$は一次独立}}$について$\qty(\mathcal{S}, \subset)$は半順序集合である。

  $\exists v \in V \setminus \qty{0_V}$について、$\exists r \in F \qty(r v = 0_V)$と仮定すると、$F$は体であることより$v = r^{-1} 0_V = 0_V$となり反する。

  ゆえに、$\qty{v} \in \mathcal{S}$\\*

  $\mathcal{S}$の全順序部分$T$を考える。

  $\forall n \in \N \forall \qty(x_m)_{m \in n} \in \qty(\bigcup T)^n$について、$\forall m \in n \exists S_m \in T \qty(x_m \in S_m)$

  $\qty{S_m \mid m \in n}$は全順序な有限集合であるので、\thmref{有限全順序集合の最大元}より最大元を持つ。

  ゆえに、$\bigcup T$は一次独立すなわち、$\bigcup T \in \mathcal{S}$

  したがって$\mathcal{S}$は帰納的である。

  \thmref{Zornの補題}より、極大元$S_0$が存在する。\\*

  $S_0$が生成系でないと仮定すると、一次独立な新たな元がとれるので、極大性に反する。
}

\lem{次元}{
  体$F$について、$F$-加群$V$を考える。$V$の任意の基底$B_1, B_2$について、全単射$f \in B_2^{B_1}$が存在する。
}{
  $P \coloneqq \qty{\varphi \in \bigcup \qty{B_2^D \mid D \in \P(B_1) \land B_1 \cap B_2 \subset D} \mid \varphi \rvert_{B_1 \cap B_2} = \id_{B_1 \cap B_2} \land C(\varphi) \coloneqq \dom(\varphi) \cup \qty(B_2 \setminus \Im(\varphi)) \text{は一次独立}}$を考える。

  ここで$p \in B_2^{B_1 \cap B_2}$について、$B_2$が基底であることから$p \in P$である。

  $P$上の半順序$\preccurlyeq$を以下のように定義する。
  \eq*{
    \varphi \preccurlyeq \psi \defiff \dom(\varphi) \subset \dom(\psi) \land \psi \rvert_{\dom(\varphi)} = \varphi
  }

  $P$の全順序部分$Q$を考える。

  自明な単射$q \in B_2^{\bigcup \qty{\dom(\varphi) \mid \varphi \in Q}}$が存在する。

  $C(q)$が一次独立でないとすると、$\exists C' \in \P(C(q)) \qty(\abs{C'} < \infty \land C' \text{は一次独立でない})$

  $\forall c \in C' \exists \varphi_c \in Q \qty(c \in \dom(\varphi_c) \cup B_2)$であり、\thmref{有限全順序集合の最大元}より、$\qty{\varphi_c \mid c \in C'} \subset Q$は最大元$\varphi_q$を持つ。

  ここで、$\Im(\varphi_q) \subset \Im(q)$であるので、$C' \subset C(\varphi_q)$であり、$C(\varphi_q)$が一次独立であることに反する。

  ゆえに$q$は上界であり、すなわち$P$は帰納的である。

  \thmref{Zornの補題}より、極大元$\sigma$が存在する。\\*

  $\exists b' \in B_2 \setminus \Im(\sigma)$であるとする。このとき$b' \notin B_1$より、$b' \notin \dom(\sigma)$である。

  $H \coloneqq \Span \qty(C(\sigma) \setminus \qty{b'})$を考える。一次独立性から$b' \notin H$

  ここで$B_1$は基底より、$\exists n \in \N \exists \qty(r_m, b_m)_{m \in \N} \in \qty(R \times B_1)^\N \qty(b' = \sum_{m \in n} r_m b_m)$と書ける。

  $b' \notin H$より、$\exists m \in n \qty(b_m \notin H)$であり、$b_m \notin \dom(\sigma)$である。

  今、以下を満たす写像$\tau \in B_2^{\dom(\sigma) \cup \qty{b_m}}$を考える。
  \eq*{
    \tau \rvert_{\dom(\sigma)} = \sigma \land \tau(b_m) = b'
  }

  $b' \notin \Im(\sigma)$より単射。$F$は体より、$C(\tau) = \qty(C(\sigma) \setminus \qty{b'}) \cup \qty{b_m}$は一次独立である。

  ゆえに$\tau \in P$であり$\sigma \prec \tau$より、$\sigma$の極大性に反する。

  背理法より$\sigma$は全射である。
  \thmref{全射と右逆写像}より、単射な右逆写像$s \in B_1^{B_2}$が存在する。\\*

  同様に、単射$t \in B_2^{B_1}$が存在する。

  \thmref{Bernsteinの定理}より、全単射$f \in B_2^{B_1}$が存在する。
}

\dfn{次元}{
  体$F$について、$F$-加群$V$について考える。

  \thmref{基底の存在}より、$V$は基底$B$を持つ。$B$が有限であるとき、\lemref{次元}より、$V$の基底の要素数は一意に定まる。
  この要素数を、$V$の次元$\dim V$と呼ぶ。
}


\lsubsection{体上の多項式}

\thmf{体上の多項式環は\textit{Euclid}整域}{体上の多項式環はEuclid整域}{
  体$F$上の多項式環$F[X]$は\textit{Euclid}整域である。
}{
  $F$が整域であることから、\corref{整域上の多項式の次数}より$F[X]$は整域である。\\*

  $f \in F[X], g \in F[X] \setminus \qty{0_{F[X]}}$を考える。

  $f = 0_{F[X]}$について、$f = 0_{F[X]} \times g + 0_{F[X]}$である。\\*

  $f \neq 0_{F[X]}$とする。

  $\deg(f) < \deg(g)$について、$f = 0_{F[X]} \times g + f$が成り立つ。\\*

  $\deg(f) \geq \deg(g)$とする。

  $\deg(f) = 0$のとき$\deg(g) = 0$より、以下の$q$について、$f = q \times g + 0_{F[X]}$である。
  \eq*{
    q(n) = f(n) g(0)^{-1}
  }

  $n \geq \deg(f)$について成り立つとき、$s(n) = \deg(f)$でも成り立つことを示す。

  以下の$h \in F[X]$を考える。
  \eq*{
    h(m) =
    \begin{cases}
      f(m) - f(\deg(f)) g(\deg(g))^{-1} g(m - \deg(f) + \deg(g)) & \qty(m \geq \deg(f) - \deg(g)) \\*
      f(m) & \qty(m < \deg(f) - \deg(g))
    \end{cases}
  }

  $h = 0_{F[X]}$のとき、以下の$q$について、$f = d g + 0_{F[X]}$で成り立つ。
  \eq*{
    d(m) =
    \begin{cases}
      f(\deg(f)) g(\deg(g))^{-1} & \qty(m = \deg(f) - \deg(g)) \\*
      0_R & \qty(m \neq \deg(f) - \deg(g))
    \end{cases}
  }

  $h \neq 0_{F[X]}$のとき、$\deg(h) \leq n$となる。帰納法の仮定より、$h = q g + r \land \deg(r) < \deg(g)$

  ゆえに、$f = d g + h = \qty(d + q) g + r \land \deg(r) < \deg(g)$

  \thmref{数学的帰納法}より成り立つ。\\*

  \corref{整域上の多項式の次数}より、$\forall f, g \in F[X] \setminus \qty{0_{F[X]}} \qty(\deg(f) \leq \deg(f g))$より成り立つ。
}

\thm{剰余の定理}{
  \thmref{体上の多項式環はEuclid整域}より、体$F$上の多項式環$F[X]$には除法が定義される。この除法は一意である。
}{
  体$F$と、$g \in F[X] \setminus \qty{0_{F[X]}}$を考える。

  $q_1 g + r_1 = q_2 g + r_2 \land \qty(r_1 = 0_{F[X]} \lor \deg(r_1) < \deg(g)) \land \qty(r_2 = 0_{F[X]} \lor \deg(r_2) < \deg(g))$とする。

  $\qty(q_1 - q_2) g = r_2 - r_1$である。

  $r_2 \neq r_1$のとき、$deg(g) > \deg(r_2 - r_1) = \deg(q_1 - q_2) + \deg(g) \geq \deg(g)$より矛盾。

  したがって$r_1 = r_2$である。整域より$q_1 = q_2$
}

\dfn{代数的}{
  体$E$と、$E$の部分体$F$、元$x \in E$について、以下を満たすとき、元$x$が$F$上代数的と呼ぶ。
  \eq*{
    \exists f \in F[X] \setminus \qty{0_{F[X]}} \qty(f(x) = 0_E)
  }
}

\thm{最小多項式}{
  体$E$と、$E$の部分体$F$、$F$上代数的な元$x \in E$について、$x$を根に持つ次数既約な多項式$f$が存在する。
}{
  以下を満たす環準同型$\varphi \in E^{F[X]}$を考える。
  \eq*{
    \varphi(g) = g(x)
  }

  ここで、$E$は体すなわち整域より、$\Ker(\varphi)$は素イデアルである。

  \thmref{体はED}と\thmref{EDはPID}より、$\aqty{f} = \Ker(\varphi)$なる多項式$f \in F[X]$が存在する。

  定義より、$\Ker(\varphi) \neq \qty{0_{F[X]}}$であるので、$f$は素元。
}

\dfn{代数的閉体}{
  体$F$上の多項式環$F[X]$を考える。

  $F[X]$の任意の次数既約な多項式$f$について、$\deg(f) = 1$であるとき、$F$を代数的閉体と呼ぶ。
}
