\lsection{距離空間}

\lsubsection{距離空間}

\dfn{距離空間}{
  空でない集合$X$と、以下を満たす写像$d \colon X \times X \rightarrow \R_{\geq 0}$について、順序対$\qty(X, d)$を距離空間と呼ぶ。
  \eqg*{
    \forall x, y \in X \qty(d(x, y) = d(y, x)) \\*
    \forall x, y \in X \qty(x = y \leftrightarrow d(x, y) = 0) \\*
    \forall x, y, z \in X \qty(d(x, z) \leq d(x, y) + d(y, z))
  }

  または単に$X$と書き、距離空間と集合どちらも表すものとする。
}

$\R$は、距離$d(x, y) \coloneqq \abs{x - y}$について、距離空間である。

\thm{距離から定まる基本近縁系}{
  距離空間$\qty(X, d)$を考える。
  以下で定める集合系$\mathcal{V}$は基本近傍系である。
  \eq*{
    \mathcal{V} \coloneqq \qty{B_r \coloneqq \qty{\qty(x, y) \in X \times X \mid d(x, y) < r} \mid r \in \R^+}
  }
}{
  第一式を示す。$\qty{\qty(x, y) \in X \times X \mid d(x, y) < 1} \in \mathcal{V}$より成り立つ。\\*

  第二式を示す。$\forall x \in X \forall r \in \R^+$を考える。\dfnref{距離空間}第二式から、$d(x, x) = 0 < r$すなわち、$\qty(x, x) \in B_r$である。\\*

  第三式を示す。$\forall r_1, r_2 \in \R^+ \qty(B_{\min \qty{r_1, r_2}} = B_{r_1} \cap B_{r_2} \in \mathcal{V})$より成り立つ。\\*

  第四式を示す。\dfnref{距離空間}第三式から、$\forall r \in \R^+ \qty(B_{r / 2} \circ B_{r / 2} \subset B_r)$である。\\*

  第五式を示す。\dfnref{距離空間}第一式から、$\forall V \in \mathcal{V} \qty(V = V^{-1})$より成り立つ。
}

\thmf{距離空間は\textit{Hausdorff}}{距離空間はHausdorff}{
  距離空間$X$は\textit{Hausdorff}である。
}{
  $\forall x, y \in X \qty(x \neq y)$を考える。

  $r \coloneqq d(x, y) / 2$について、\dfnref{距離空間}第二式より、$r \in \R^+$である。

  $B_r[x] \cap B_r[y] = \varnothing$より、\textit{Hausdorff}である。
}
