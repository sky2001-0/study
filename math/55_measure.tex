\lsection{測度論}

\lsubsection{測度}

\dfnf{$\sigma$-加法族}{sigma-加法族}{
  空でない集合$X$について、以下を満たす集合系$\mathcal{S} \in 2^{2^X}$を$\sigma$-加法族と呼ぶ。
  \eqg*{
    X \in \mathcal{S} \\*
    \forall S \in \mathcal{S} \qty(X \setminus S \in \mathcal{S}) \\*
    \forall A \in 2^\mathcal{S} \qty(\text{$A$は可算} \rightarrow \bigcup A \in \mathcal{S})
  }
}

\lem*{
  空でない集合$X$上の$\sigma$-加法族$\mathcal{S}$について、以下が成り立つ。
  \eqg*{
    \varnothing \in \mathcal{S} \\*
    \forall S_1, S_2 \in \mathcal{S} \qty(S_1 \setminus S_2 \in \mathcal{S}) \\*
    \forall A \in 2^\mathcal{S} \qty(\text{$A$は可算} \rightarrow \bigcap A \in \mathcal{S})
  }
}{
  第一式を示す。\dfnref{sigma-加法族}第一式、第二式より$\varnothing = X \setminus X \in \mathcal{S}$\\*

  第二式を示す。\dfnref{sigma-加法族}第二式、第三式より$S_1 \setminus S_2 = X \setminus \qty(\qty(X \setminus S_1) \cup S_2) \in \mathcal{S}$\\*

  第三式を示す。$A = \varnothing$のとき$\bigcap A = \varnothing$よりすでに示した第一式より成り立つ。

  $A \neq \varnothing$のとき、\thmref{De Morganの法則}と\dfnref{sigma-加法族}第二式、第三式より成り立つ。
}

\dfn{可測空間}{
  空でない集合$X$と、$X$上の$\sigma$-加法族$\mathcal{S}$について、順序対$\qty(X, \mathcal{S})$を可測空間と呼ぶ。
  または単に$X$と書き、可測空間と集合どちらも表すものとする。
}

\dfn{無限を入れた実数}{
  順序体$\R$について、全順序と可換群の構造を備えた集合$A \coloneqq \R \cup \qty{\infty}$を以下のように定義する。
  \eqg*{
    \forall a \in A \qty(a \leq \infty) \\*
    \forall a \in A \qty(a + \infty = \infty)
  }
}

\dfn{測度}{
  可測空間$\qty(X, \mathcal{S})$について、以下を満たす写像$\mu \colon \mathcal{S} \rightarrow \R_{\geq 0} \cup \qty{\infty}$を測度と呼ぶ。
  \eqg*{
    \mu(\varnothing) = 0 \\*
    \forall A \in 2^\mathcal{S} \qty(\text{$A$は可算} \land \rightarrow \bigcap A \in \mathcal{S})
  }
}

\begin{comment}

\dfnf{\textit{Borel}集合族}{Borel集合族}{
  位相空間$\qty(X, \mathcal{O})$について、以下で定める集合$\mathcal{B}(X, \mathcal{O})$を\textit{Borel}集合族と呼ぶ。
  \eq*{
    \mathcal{B}(X, \mathcal{O}) = \bigcap \qty{\mathcal{S} \in 2^{2^X} \mid \text{$\mathcal{S}$は$\sigma$-加法族} \land \mathcal{O} \subset \mathcal{S}}
  }
}

\thm*{
  \textit{Borel}集合族は$\sigma$-加法族である。
}{}

\end{comment}

